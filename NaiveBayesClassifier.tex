\documentclass[12pt]{article}

%opening
\title{Coursera - Machine Learning}
\author{www.Stats-Lab.com}

\begin{document}

\maketitle


\subsection{Naive Bayes Classifier}


\begin{itemize}
\item A Naive Bayes classifier is a simple probabilistic classifier based on applying Bayes' theorem with strong (naive) independence assumptions.\item  A more descriptive term for the underlying probability model would be "independent feature model".
\item  In simple terms, a naive Bayes classifier assumes that the presence (or absence) of a particular feature of a class is unrelated to the presence (or absence) of any other feature, given the class variable. 
\item  For example, a fruit may be considered to be an apple if it is red, round, and about 4" in diameter. 
\item  Even if these features depend on each other or upon the existence of the other features, a naive Bayes classifier considers all of these properties to independently contribute to the probability that this fruit is an apple
\end{itemize}
\end{document}
