\documentclass[]{article}

\begin{document}



\section{Principal Component Analysis}

\textbf{Recommended applications of PCA}
\begin{itemize}
\item \textbf{Data visualization:} Reduce data to 2D (or 3D) so that it can be plotted.

This is a good use of PCA, as it can give you intuition about your data that would otherwise be impossible to see.

\item \textbf{Data compression:} Reduce the dimension of your data, so that it takes up less memory / disk space.  

If memory or disk space is limited, PCA allows you to save space in exchange for losing a little of the data's information. This can be a reasonable tradeoff.
\end{itemize}
\textbf{Inappropriate applications of PCA}
\begin{itemize}
\item \textbf{Data visualization:} To take 2D data, and find a different way of plotting it in 2D (using k=2).
		
 
 You should use PCA to visualize data with dimension higher than 3, not data that you can already visualize.
\item To get more features to feed into a learning algorithm.	 

	PCA will reduce the number of features, not expand it.
\end{itemize}
\end{document}
