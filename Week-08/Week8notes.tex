%------------------------------------------------%

%ML Week 8

\subsection*{Finding Closest Centroids}

In the cluste assignment phase of the algorithm, the algorithm assigns every training example $x^{(n)}$ to the closest 
centroid, given the current position of the centroids.

\[ C^{(i)} := J \mbox{ that minimizes } \| x^{} - \mu_j \| ^2 \]

\begin{itemize}
\item $C^{(i)}$ index of the centroid clostest to $x^{(i)}$
\item $\mu_j$ is the position of the $j-$th centroid.
\end{itemize}

Computing Centroid Means

\[ 
\mu_k := \frac{1}{abs(C_k)} \sum_{i \in C_k} x^{(i)}
\]
%---------------%

A good way to initialize k-mean is to select $k$ distinct examples from the training set and set the cluster centroids equal to these selected examples.

On every iteration of $k-$means, the cost function $J(C^{(1)},C^{(2)},\ldots, C^{(n)},
\mu_1,\ldots \mu_k)$

The distortion function should either stay the same or decrease, in particular it should not increase.

%---------------%
Recommended Applications of PCA

\begin{itemize}
\item Data Compression: reducing the dimensionof input data $x^{(i)}$, which will be used in a supervised learning algorithm
(i.e. use PCA data so that your supervised learning algorithm runs faster.

\item Data Visualization: Reduce data to 2D ( or 3D) so that it can be plotted.
\end{itemize}
