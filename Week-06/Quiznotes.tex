
%------------------------------------------------%

%ML Week 6

\subsection*{High variance}

\begin{itemize}
\item indicated by gap in errors between training and testing data sets.
\item Algorithm has overfit the data for the training data set.
\item Increasing the regularization parameter will reduce overfitting
\end{itemize}
The recommended way to choose a value of regularization parameter $\lambda$ to use is to choose
the lowest cross validation error.
You should not use the training data set for this purpose.
Close
Advice for Applying Machine Learning

5 questions
1
point
1. 
You train a learning algorithm, and find that it has unacceptably high error on the test set. You plot the learning curve, and obtain the figure below. Is the algorithm suffering from high bias, high variance, or neither?



High bias

Neither

CORRECT High variance 
%================================================%
2. 
Suppose you have implemented regularized logistic regression

to classify what object is in an image (i.e., to do object

recognition). However, when you test your hypothesis on a new

set of images, you find that it makes unacceptably large

errors with its predictions on the new images. However, your

hypothesis performs well (has low error) on the

training set. Which of the following are promising steps to

take? Check all that apply.

WRONG Try adding polynomial features.

CORRECT Get more training examples.

WRONG Use fewer training examples.

CORRECT Try using a smaller set of features.
%================================================%
3. 
Suppose you have implemented regularized logistic regression

to predict what items customers will purchase on a web

shopping site. However, when you test your hypothesis on a new

set of customers, you find that it makes unacceptably large

errors in its predictions. Furthermore, the hypothesis

performs poorly on the training set. Which of the

following might be promising steps to take? Check all that

apply.


WRONG Try evaluating the hypothesis on a cross validation set rather than the test set.


CORRECT  Try adding polynomial features.

WRONG Use fewer training examples.

CORRECT Try decreasing the regularization parameter λ.
%================================================%
4. 
Which of the following statements are true? Check all that apply.

CORRECT Suppose you are using linear regression to predict housing prices, and your dataset comes sorted in order of increasing sizes of houses. It is then important to randomly shuffle the dataset before splitting it into training, validation and test sets, so that we don’t have all the smallest houses going into the training set, and all the largest houses going into the test set.

Suppose you are training a logistic regression classifier using polynomial features and want to select what degree polynomial (denoted d in the lecture videos) to use. After training the classifier on the entire training set, you decide to use a subset of the training examples as a validation set. This will work just as well as having a validation set that is separate (disjoint) from the training set.

CORRECT A typical split of a dataset into training, validation and test sets might be 60% training set, 20% validation set, and 20% test set.

It is okay to use data from the test set to choose the regularization parameter λ, but not the model parameters (θ).

%================================================%
5. 
Which of the following statements are true? Check all that apply.

CORRECT When debugging learning algorithms, it is useful to plot a learning curve to understand if there is a high bias or high variance problem.

We always prefer models with high variance (over those with high bias) as they will able to better fit the training set.

CORRECT If a learning algorithm is suffering from high variance, adding more training examples is likely to improve the test error.

CORRECT If a learning algorithm is suffering from high bias, only adding more training examples may not improve the test error significantly.
Submit Quiz


