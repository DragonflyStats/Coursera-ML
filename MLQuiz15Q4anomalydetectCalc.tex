\documentclass[]{article}

\begin{document}



\section*{Anomaly Detection}
\begin{itemize}
\item Suppose you are developing an anomaly detection system to catch manufacturing defects in airplane engines. 
\item Your model uses
{ 
\Large
\[p(x)= \Pi ^{n}_{j=1} p(x_j;\mu_j,\sigma^2_j)\] 
}
\item You have two features $x_1$ = \textit{\textbf{vibration intensity}}, and $x_2$ = \textit{\textbf{heat generated}}. 
\item Both $x_1$ and $x_2$ take on values between 0 and 1 (and are strictly greater than 0), and for most "normal" engines you expect that $x_2 \approx x_2$. 
\item One of the suspected anomalies is that a flawed engine may vibrate very intensely even without generating much heat (large $x_1$, small $x_2$), 
even though the particular values of $x_1$ and $x_2$ may not fall outside their typical ranges of values. 
\item What additional feature $x_3$ should you create to capture these types of anomalies:
\end{itemize}
\textbf{Solution Options}
\begin{itemize}
\item $x_3=x_1+x_2$	This could take on large or small values for both normal and anomalous examples, so it is not a good feature.
\end{itemize}
\end{document}